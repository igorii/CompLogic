\documentclass[12pt]{article}
\usepackage[dvipsnames]{xcolor}
\usepackage{fancybox}
\usepackage{fancyvrb}
\usepackage{graphicx}
\usepackage{amssymb}
\usepackage{amsmath}
\usepackage{epsfig}
\usepackage{color}
\usepackage{multicol}
\usepackage{hhline}
\usepackage{xspace,epic,eepic,graphicx}
\usepackage{latexsym}

\RecustomVerbatimCommand{\VerbatimInput}{VerbatimInput}%
{fontsize=\footnotesize,
 %
 frame=lines,  % top and bottom rule only
 framesep=2em, % separation between frame and text
 rulecolor=\color{Gray},
 label=\fbox{\color{Black}data.txt},
 labelposition=topline,
}

\newcommand{\mc}[1]{\mathcal{ #1}}
\newcommand{\e}[1]{\emph{#1}}
\newcommand{\ignore}[1]{}
\newcommand{\boxtheorem}{\hfill $\Box$\\}
\newcommand{\nit}[1]{{\it #1}}

\begin{document}
\thispagestyle{empty}

\vspace*{-3.5cm}
\begin{center} \bf \large COMP 4900C~ Computational Logic and Automated Reasoning\\ Winter 2015~~ Assignment I
\end{center}

{\small \noindent {Timothy Thornton}\\
{\small \noindent {100822254}\\
{\small \noindent {tim.thornton@cmail.carleton.ca}\\

{\small \noindent {\bf Question 0}\\

\noindent 0. \ Install ``Prover 9" and Latex in your computer. In your solutions include the code and run for/of whichever system is used. For this, the ``verbatim" Latex environment is
useful. 
\\

Version information reported by Prover9:

\begin{verbatim}
Prover9 (32) version Oct-2007, Oct 2007.
\end{verbatim}

{\small \noindent {\bf Question 1}\\

\noindent 1. \ (lecture 1, slide 17) \ Rerun in Prover 9 the superman example.\\

The following code represents the Superman example. The atomic constants have been renamed to avoid interpretation as variables.

\VerbatimInput[label=superman.in]{sources/superman.in}

Output of the above input:

\VerbatimInput[label=superman.out]{sources/superman.out}

\newpage
2. \ (lecture 3, slide 41) \ Rerun in Prover 9 the group example.\\

The following code represents the Group example.

\VerbatimInput[label=group.in]{sources/group.in}

Output of the above input:

\VerbatimInput[label=group.out]{sources/group.out}

\newpage
3. \ (lecture 2, slide 33. \ Prove with Prover 9 the theorem about Boolean algebras.\\

The following code proves idempotency of Boolean algebras.

\VerbatimInput[label=boolean.in]{sources/boolean.in}

Output of the above input:

\VerbatimInput[label=boolean.out]{sources/boolean.out}

\end{document}
